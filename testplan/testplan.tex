\documentclass[12pt,a4paper]{article}
%\usepackage{bookman}
%\usepackage{newcent}
%\usepackage{palantino}
%\usepackage{times}
\title{Beta-Test pyssm 0.2}
\usepackage{bera}
\usepackage{ngerman}
\usepackage[T1]{fontenc}
\usepackage{wasysym}
\setlength{\parindent}{0cm}
\setlength{\parskip}{1ex}
\usepackage{listings}
\usepackage{hyperref}
\newcommand{\pyssmtest}[2]{
\subsection{#1}
	Maximal erlaubte Fehler: #2\\
	\Square \hspace{0.2cm} Bestanden beim ersten Versuch\\
	\Square \hspace{0.2cm} Bestanden beim zweiten Versuch\\
	\Square \hspace{0.2cm} nicht bestanden nach zweitem Versuch\\
	\TextField[width=15cm,multiline]{Aufgetretene Fehler:\\}\\
	\TextField[width=15cm,multiline]{Sonstige Anmerkungen:\\}\\}


\begin{document}
\begin{Form}
{\Huge Beta-Test pyssm 0.2}\vspace{0.5cm}\\
Bei Fragen und Anmerkungen stehe ich �ber bjoern.guth@rwth-aachen.de jeder Zeit zur Verf�gung.\\\\
Kontakt f�r evtl. R�ckfragen:\\
\TextField[name=Name,width=10cm]{Name:}\\
\TextField[name=Name,width=10cm]{Mail:\hspace{.3cm}}\\

\setcounter{section}{-3}
\section{Installation}
	pyssm muss in folgender Reihenfolge installiert werden:
	\begin{enumerate}
		\item python-2.7.2.msi \\ Installationspfad notieren, da der sp�ter noch gebraucht wird.\vspace{.5cm}\\\underline{\hspace{15cm}}
		\item pygame-1.9.1.win32-py2.7.msi
		\item pyssm-0.2.win32.exe
	\end{enumerate}

\section{Starten von pyssm vom Terminal}
Um die von python ausgegebenen Fehlermeldungen ausgegeben zu bekommen, muss man pyssm von einem Terminal aus starten. Gehe hierf�r einfach die folgenden Schritte durch:
\begin{itemize}
	\item Klicke auf Start und start cmd.exe, indem du in der Suchzeile
		\begin{lstlisting}
			cmd.exe
		\end{lstlisting}
		eingibst. 
	\item �ndere den aktuellen Ordner, indem du
	\begin{lstlisting}
		cd <Installationspfad>\Scripts
	\end{lstlisting}
	eingibst. Ersetze <Installationspfad> durch den Pfad, den du w�hrend der Installation notiert hast.
	\item Jetzt kannst du 	pyssm durch die Engabe von
		\begin{lstlisting}
			pyssm.py
		\end{lstlisting}
	starten.
	\item Um ein bestimmtes Modul zu starten, kannst du pyssm.py mit der Option -s und dem Buchstaben des Moduls starten. Der Buchstabe des Moduls muss als Kleinbuchstabe eingegeben werden! (um Modul Z zu starten, m�sstest du also pyssm.py -s z eingeben.
\end{itemize}
	cmd l�uft so auch weietr, nachden pyssm aus welchen Gr�nden auch immer beendet wird.

\section{Interpretation von Fehlermeldungen}
	Um einen Bug m�glichst schnell beheben zu k�nnen, brauche ich ein paar Informationen �ber den aufgetretenen Fehler. Diese k�nnen den Fehlermeldungen, den s.g. Exceptions, entnommen werden. Eine Exception gibt einen Text im Terminal aus, der wie folgt ausschaut:
	\begin{lstlisting}
	Traceback (most recent call last):
	  File "`./pyssm.py"', line 126, in <module>
	    stage = Stage_Q(log,teach,test,rep,neo)
	  File "`/home/bjoern/hiwi/pyssm/level/stages.py"', line 852, in __init__
	    stage = Balloon_Engine(log_q,self.get_path('data/modul_q',neo=neo))
	TypeError: get_path() got an unexpected keyword argument 'neo'
	\end{lstlisting}
	Wichtig sind davon f�r mich zum einen die letzte Zeile, die mit File beginnt, da diese mir sagt, an welcher Stelle in welcher Datei ein Fehler aufgetreten ist, und die letzte Zeile, da hier der Fehler selbst beschrieben wird.

\section{Modul A}
\pyssmtest{KO}{2}
\pyssmtest{ME}{2}
\pyssmtest{Test Modul A}{2}

\section{Modul F}
\pyssmtest{A}{2}
\pyssmtest{I}{2}
\pyssmtest{Test Modul F}{2}

\section{Modul U}
\pyssmtest{Test Modul U}{4}

\section{Modul P}
\pyssmtest{RI}{2}
\pyssmtest{Test Modul P}{4}

\section{Modul L}
\pyssmtest{O}{2}
\pyssmtest{Test Modul O}{3}

\section{Modul Q}
\pyssmtest{FA}{2}
\pyssmtest{Test Modul Q}{4}

\section{Modul B}
\pyssmtest{E}{2}
\pyssmtest{Test Modul B}{4}

\section{Modul Z}
\pyssmtest{Test Modul Z}{5}
\end{Form}
\end{document}
